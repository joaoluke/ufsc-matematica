\documentclass[12pt,a4paper]{article}

\usepackage[utf8]{inputenc}
\usepackage[T1]{fontenc}
\usepackage[brazil]{babel}
\usepackage{geometry}
\usepackage{graphicx}
\usepackage{tikz}
\graphicspath{{../../../../../assets/}}
\geometry{a4paper, margin=2.5cm}
\usepackage{setspace}
\usepackage{amsmath, amssymb}
\usepackage{enumitem}

\newcommand{\universidade}{Universidade Federal de Santa Catarina}
\newcommand{\centro}{Centro de Ciências Físicas e Matemáticas}
\newcommand{\curso}{Curso de Matemática - Licenciatura}
\newcommand{\disciplina}{Geometria I}
\newcommand{\professor}{Prof. Marcelo Sobottka}
\newcommand{\tutor}{Prof. Marcos Fabiano Firbida Eduardo}
\newcommand{\autor}{João Lucas de Oliveira}
\newcommand{\dataentrega}{Setembro de 2025}

\begin{document}

\begin{center}
    \includegraphics[width=3cm]{ufsc_logo}\\[0.3cm]
    \textbf{\universidade}\\
    \centro\\
    \curso\\[1cm]
    \disciplina{}\\
    \textbf{Lista II} -- Axiomas de Incid\^encia, Ordem, Medi\c{c}\~ao de Segmento, Medi\c{c}\~ao de \^angulo\\[0.5cm]
    \textbf{Professor:} \professor{}\\
    \textbf{Tutor:} \tutor{}\\
    \textbf{Aluno:} \autor{}\\
    \textbf{Data:} \dataentrega{}
\end{center}

\vspace{1cm}

\section*{Questão 1 -- Para cada caso abaixo, indique quais dos axiomas são (ou podem ser) satisfeitos e quais não podem ser satisfeitos:}
    \begin{figure}[h]
        \centering
        \begin{tikzpicture}[scale=3, line cap=round, line join=round]
        \def\R{1}
        \def\a{0.5}
        \def\phi{35}
        \draw[thick] (-\R,0) arc[start angle=180, end angle=0, radius=\R];
        \draw[dotted, thick, domain=0:360, samples=200]
            plot ({\R*cos(\x)}, {\a*\R*sin(\x)});
        \begin{scope}[rotate=\phi]
            \draw[red, very thick, domain=-60:60, samples=100]
            plot ({0.98*\R*cos(\x)}, {0.55*\R*sin(\x)});
        \end{scope}
        \fill[red] (-0.80,0) circle(0.015);
        \fill[red] (0.40,0.72) circle(0.015);
        \end{tikzpicture}
        \caption{Semi-esfera, equador pontilhado e reta (arco de círculo máximo) em vermelho.}
    \end{figure}
  
    \begin{enumerate}[label= (\roman*)]
        \item  $A < B$.

        \item  $A > B$.

        \item  $A \not< B$.

        \item  $A \not> B$.

    \end{enumerate}

\end{document}
